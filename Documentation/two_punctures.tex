%****** Start of file apssamp.tex ******
%
%   This file is part of the APS files in the REVTeX 4.1 distribution.
%   Version 4.1r of REVTeX, August 2010
%
%   Copyright (c) 2009, 2010 The American Physical Society.
%
%   See the REVTeX 4 README file for restrictions and more information.
%
% TeX'ing this file requires that you have AMS-LaTeX 2.0 installed
% as well as the rest of the prerequisites for REVTeX 4.1
%
% See the REVTeX 4 README file
% It also requires running BibTeX. The commands are as follows:
%
%  1)  latex apssamp.tex
%  2)  bibtex apssamp
%  3)  latex apssamp.tex
%  4)  latex apssamp.tex
%
\documentclass[%
%reprint,
%preprint,
notitlepage,
%superscriptaddress,
%groupedaddress,
%unsortedaddress,
%runinaddress,
%frontmatterverbose, 
report,
%showpacs,preprintnumbers,
nofootinbib,
%nobibnotes,
%bibnotes,
 amsmath,amssymb,
 aps,
%pra,
%prb,
%rmp,
%prstab,
%prstper,
%floatfix,
%]{revtex4}
]{revtex4-1}

\usepackage{graphicx}
%\graphicspath{{.}{jripley_plots/}}
% Include figure files
%\usepackage{dcolumn}% Align table columns on decimal point
\usepackage{float}% to force figure placement 
\usepackage{bm}% bold math
\usepackage[hidelinks]{hyperref}
%\usepackage[mathlines]{lineno}% Enable numbering of text and display math
%\linenumbers\relax % Commence numbering lines
\usepackage{color}

\usepackage[utf8]{inputenc}
\usepackage[T1]{fontenc}
\usepackage{slashed}

%\usepackage{fullpage}
\usepackage{enumerate}

\usepackage{mathalfa,amssymb,amsthm}
\usepackage{amsmath}

%\usepackage{subfig}

\newcommand{\we}[1]{\textcolor{blue}{WE: #1}}
\newcommand{\jr}[1]{\textcolor{green}{JR: #1}}

%\newcommand{\ssec}[1]{{{\em #1.}---}}
\newcommand{\ssec}[1]{\section{#1}}

%\usepackage[showframe,%Uncomment any one of the following lines to test 
%%scale=0.7, marginratio={1:1, 2:3}, ignoreall,% default settings
%%text={7in,10in},centering,
%%margin=1.5in,
%%total={6.5in,8.75in}, top=1.2in, left=0.9in, includefoot,
%%height=10in,a5paper,hmargin={3cm,0.8in},
%]{geometry}

\begin{document}

%\preprint{APS/123-QED}
%=============================================================================
\title{Binary black hole initial data for $4\partial ST$ gravity}
%\thanks{A footnote to the article title}%
%-----------------------------------------------------------------------------
\author{William E. East}
\email{weast@perimeterinstitute.ca}
\affiliation{%
Perimeter Institute for Theoretical Physics, Waterloo, Ontario N2L 2Y5, Canada.
}%
\author{Justin L. Ripley}
\email{lloydripley@gmail.com}
\affiliation{%
DAMTP,
Centre for Mathematical Sciences,
University of Cambridge,
Wilberforce Road, Cambridge CB3 0WA, UK.
}%
%=============================================================================
\begin{abstract}
We study the dynamics of binary black hole systems for a class of
scalar-tensor theories.
We construct an initial value problem for the theory.
\end{abstract}
%=============================================================================
\maketitle
%=============================================================================
\ssec{Introduction}%
We use units with $G=c=1$, with $-+++$ metric signature, and
$R^{\mu}{}_{\alpha\nu\beta}=\partial_{\nu}\Gamma^{\mu}_{\alpha\beta}-\cdots$.
Lower-case Greek letters label spacetime indices, while lower-case
Latin letter label spatial indices.
%=============================================================================
\ssec{Equations of motion}%
Here we briefly review the theories we consider and the arguments
behind why black holes may be dynamically unstable to scalar
field perturbations in these theories.

The action for $4\partial ST$ gravity is
\begin{align}
\label{eq:leading_order_in_derivatives_action}
    S
    =
    \frac{1}{8\pi}
   \int d^4x\sqrt{-g}
   \Big(&
   \frac{1}{2}R
   +
   X
   -
   V\left(\phi\right)
   \nonumber\\
   &
   +
   \alpha\left(\phi\right)X^2
   +	
   \beta\left(\phi\right)\mathcal{G}
   \Big)
    ,
\end{align}
where $X$ is the scalar field kinetic term
and $\mathcal{G}$ is the Gauss-Bonnet scalar:
\begin{subequations}
\begin{align}
   X
   \equiv&
   -
   \frac{1}{2}g^{\mu\nu}\nabla_{\mu}\phi\nabla_{\nu}\phi
   ,\\
   \mathcal{G}
   \equiv&
   -
   \frac{1}{4}
   \delta^{\alpha\beta\gamma\delta}_{\mu\nu\rho\sigma}
   R^{\mu\nu}{}_{\alpha\beta}
   R^{\rho\sigma}{}_{\gamma\delta}
   ,
\end{align}
\end{subequations}
(here $\delta^{\alpha\beta\gamma\delta}_{\mu\nu\rho\sigma}$
is the generalized Kronecker delta).
Varying Eq.~\eqref{eq:leading_order_in_derivatives_action}
with respect to the scalar and metric fields gives us
\begin{align}
\label{eq:eom_edgb_scalar}
   E^{(\phi)}
   &\equiv
   \Box\phi
   -  
   V^{\prime}\left(\phi\right)
   \nonumber\\&
   +  
   2\alpha\left(\phi\right)X \Box\phi
   -  
   2\alpha\left(\phi\right)
   \nabla^{\alpha}\phi\nabla^{\beta}\phi\nabla_{\alpha}\nabla_{\beta}\phi
   \nonumber\\&
   -  
   3\alpha^{\prime}\left(\phi\right)X^2
   +  
   \beta^{\prime}\left(\phi\right)\mathcal{G}
   =
   0
   ,\\
\label{eq:eom_edgb_tensor}
   E^{(g)}_{\alpha\beta}
   &\equiv
   R_{\alpha\beta}
   -  
   \frac{1}{2}g_{\alpha\beta}R
   \nonumber\\&
   -  
   \nabla_{\alpha}\phi\nabla_{\beta}\phi
   +  
   \left(-X+V\left(\phi\right)\right)g_{\alpha\beta}
   \nonumber\\&
   -  
   2\alpha\left(\phi\right)X\nabla_{\alpha}\phi\nabla_{\beta}\phi
   -  
   \alpha\left(\phi\right)X^2g_{\alpha\beta}
   \nonumber\\&
   +  
   2\delta^{\mu\nu\gamma\delta}_{\sigma\rho\eta(\alpha}
   g_{\beta)\delta}R^{\sigma\rho}{}_{\mu\nu}
   \nabla^{\eta}\nabla_{\gamma}\beta\left(\phi\right) 
   =
   0
   .
\end{align}
%=============================================================================
\ssec{Puncture-type binary black hole initial data}%
On our initial data surface we must satisfy the generalizations of
the Hamiltonian $\mathcal{H}$ and momentum $\mathcal{M}$ constraints:
\label{eq:constraints}
\begin{align}
\label{eq:hamiltonian_constraint}
   \mathcal{H}
   \equiv&
   n_{\alpha}n^{\beta}E^{(g)}_{\alpha\beta}
   \nonumber\\
   =&
   \frac{1}{2}h^{\alpha\beta}h^{\gamma\delta}R_{\alpha\gamma\beta\delta}
   \nonumber\\&
   -
   \frac{1}{2}h^{\alpha\beta}D_{\alpha}\phi D_{\beta}\phi
   -
   V\left(\phi\right)
   \nonumber\\&
   -  
   2\alpha\left(\phi\right)X\left(n^{\alpha}\nabla_{\alpha}\phi\right)^2
   +  
   \alpha\left(\phi\right)X^2
   \nonumber\\&
   +  
   2n^{\alpha}n_{\delta}
   \delta^{\mu\nu\gamma\delta}_{\rho\sigma\eta\alpha}
   R^{\rho\sigma}{}_{\mu\nu}\nabla^{\eta}\nabla_{\gamma}\beta\left(\phi\right)
   \\
\label{eq:momentum_constraint}
   \mathcal{M}_{\lambda}
   \equiv&
   n_{\alpha}h^{\beta}{}_{\lambda}E^{(g)}_{\alpha\beta}
   \nonumber\\
   =&
   n^{\alpha}h^{\beta}{}_{\lambda}R_{\alpha\beta}
   \nonumber\\&
   -  
   n^{\alpha}\nabla_{\alpha}\phi D_{\lambda}\phi
   \nonumber\\&
   -  
   2\alpha\left(\phi\right)X
   n^{\alpha}\nabla_{\alpha}\phi D_{\lambda}\phi
   \nonumber\\&
   +  
   2n_{\alpha}h^{\beta}{}_{\lambda}
   \delta^{\mu\nu\gamma\alpha}_{\rho\sigma\eta\beta}
   R^{\rho\sigma}{}_{\mu\nu}
   \nabla_{\gamma}\nabla^{\eta}\beta\left(\phi\right)
   .
\end{align}

Before presenting the general equations one needs to solve,
we not that if we choose $\phi=\partial_t\phi=V=0$ on the initial data slice,
then the constraint equations reduce to those of vacuum GR\cite{East:2020hgw}.
In this case, binary black hole initial data solves the constraint
equations for $4\partial ST$ gravity,
and we could use the \texttt{twopunctures} code as-is.
Solving for more general initial data requires us to solve
the full constraint equations.

When $\phi$ and $\partial_t\phi$ are not equal to zero,
the constraint equations remain elliptic equations
(i.e. they do not involve
second time derivatives of the metric and scalar field;
for a derivation of this fact see
Appendix~\ref{eq:conformal_decomp_derivation}).
%=============================================================================
\ssec{Methodology}%

We numerically evolve the full ESBG equations of motion (EOM)
using the modified generalized harmonic
formulation~\cite{Kovacs:2020pns,Kovacs:2020ywu}
as described in Ref.~\cite{East:2020hgw}.
We use similar choices for the gauge, numerical parameters, etc. as in
Ref.~\cite{East:2020hgw},
except that we find the scalarized BHs evolved here also benefit from 
the addition of long wavelength constraint damping obtained by setting
$\rho=-0.5$ in Eq.~(2) of Ref.~\cite{East:2020hgw}.
%=============================================================================
\ssec{Results}
   Binary black hole waveforms? $r\times\psi_4$ and $r\times\phi$?
%=============================================================================
\ssec{conclusion}%

\ssec{acknowledgements}%
w.e. acknowledges support from an nserc discovery grant.
this research was
supported in part by perimeter institute for theoretical physics.  research at
perimeter institute is supported by the government of canada through the
department of innovation, science and economic development canada and by the
province of ontario through the ministry of research, innovation and science.
this research was enabled in part by support provided by scinet
(www.scinethpc.ca/) and compute canada (www.computecanada.ca).
some of the simulations presented in this article were performed on
computational resources managed and supported by princeton research computing,
a consortium of groups including the princeton institute for
computational science and engineering (picscie)
and the office of information technology's high performance
computing center and visualization laboratory at princeton university.

\bibliographystyle{apsrev4-1.bst}
\bibliography{../mod_grav}

\appendix
%=============================================================================
\section{Derivation of conformal decomposition of the Hamiltonian
and momentum constraint equations}
\label{eq:conformal_decomp_derivation}
%=============================================================================
\subsection{General equations}
\label{eq:general_id_eqns}

   Here we review York's transverse-traceless procedure to construct
maximal, conformally flat initial data, and derive the constraint
equations for $4\partial ST$ gravity in that formalism.

We begin by foliating the spacetime with spatial sections with unit
normal vectors $n^{\alpha}$. We split up the metric as
\begin{align}
   g^{\alpha\beta}
   =
   -
   n^{\alpha}n^{\beta}
   +
   h^{\alpha\beta}
   .
\end{align}
We next perform the usual $1+3$ dimensional split of the spacetime metric:
\begin{align}
   g_{\alpha\beta}dx^{\alpha}dx^{\beta}
   =
   -
   N^2dt^2
   +
   h_{ij}
   \left(dx^i+N^tdt\right)\left(dx^j+N^jdt\right)
   .
\end{align}
The intrinsic metric to each spatial section is $h_{ij}$,
$N$ is the lapse, and $N^i$ is the shift. The spatial
sections have normal gradient $n_{\alpha}=(-N,0,0,0)$.
The extrinsic curvature to the spatial sections is
\begin{subequations}
\begin{align}
   K_{\alpha\beta}
   \equiv&
   -
   h_{\alpha}{}^{\mu}h_{\beta}{}^{\nu}\nabla_{\mu}n_{\nu}
   ,\\
   \implies
   K_{ij}
   =&
   -
   \frac{1}{2N}
   \left(
      \partial_th_{ij}
      -
      D_iN_j
      -
      D_jN_i
   \right)
   .
\end{align}
\end{subequations}
We will also make use of the notation
\begin{subequations}
\begin{align}
   a_{\alpha}
   \equiv&
   n^{\beta}\nabla_{\beta}n_{\alpha}
   ,\\
   K_{\phi}
   \equiv&
   -\frac{1}{2}\mathcal{L}_n\phi
   =
   -\frac{1}{2}n^{\alpha}\nabla_{\alpha}\phi
   .
\end{align}
\end{subequations}
Under the $1+3$ decomposition the ESGB contribution
to the Hamiltonian and momentum constraint is
\begin{align}
\label{eq:hamiltonian_constraint_3p1}
   \mathcal{H}^{(GB)}
   =&
   2n^{\alpha}n_{\delta}
   \delta^{\mu\nu\gamma\delta}_{\rho\sigma\eta\alpha}
   R^{\rho\sigma}{}_{\mu\nu}
   \nabla^{\eta}\nabla_{\gamma}\beta\left(\phi\right)
   ,\nonumber\\
   =&
   2n^{\alpha}n_{\delta}
   \delta^{\mu\nu\gamma\delta}_{\rho\sigma\eta\alpha}
   \left(
      h^{\rho\rho'}h^{\sigma\sigma'}
      h^{\mu'}{}_{\mu}h^{\nu'}{}_{\nu}
      R_{\rho'\sigma'\mu'\nu'}
   \right)
   \left(
      h^{\eta\eta'}h^{\gamma'}{}_{\gamma}
      \nabla_{\eta'}\nabla_{\gamma'}\beta\left(\phi\right)
   \right)
   ,\nonumber\\
   =&
   2n^{\alpha}n_{\delta}
   \delta^{\mu\nu\gamma\delta}_{\rho\sigma\eta\alpha}
   \left(
      {}^{(3)}R^{\rho\sigma}{}_{\mu\nu}
      +
      2K^{\rho}{}_{\mu}K^{\sigma}{}_{\nu}
   \right)
   \left(
      D^{\eta}D_{\gamma}\beta\left(\phi\right)
      +
      K^{\eta}{}_{\gamma}B\left(\phi\right)
   \right)
   ,\nonumber\\
   =&
   -2
   \delta^{mng}_{rse}
   \left(
      {}^{(3)}R^{rs}{}_{mn}
      +
      2K^{r}{}_{m}K^{s}{}_{n}
   \right)
   \left(
      D^{e}D_{g}\beta\left(\phi\right)
      +
      K^{e}{}_{g}B\left(\phi\right)
   \right)
   .
\end{align}
The ESGB contribution to the momentum constraint is more complicated:
\begin{align}
   \label{eq:momentum_constraint_3p1}
   \mathcal{M}^{(GB)}_{\lambda}
   =&
   2n_{\alpha}h^{\beta}{}_{\lambda}
   \delta^{\mu\nu\gamma\alpha}_{\rho\sigma\eta\beta}
   R^{\rho\sigma}{}_{\mu\nu}
   \nabla_{\gamma}\nabla^{\eta}\beta\left(\phi\right)
   ,\\
   =&
   -
   2n_{\alpha}h^{\beta}{}_{\lambda}
   \delta^{\mu\nu\gamma\alpha}_{\rho\sigma\eta\beta}
   \Big(
      \left(
         -
         2n^{\rho}n_{\rho'}h^{\sigma}{}_{\sigma'}
         h_{\mu}{}^{\mu'}h_{\nu}{}^{\nu'}
         R^{\rho'\sigma'}{}_{\mu'\nu'}
      \right)
      \left(
         h_{\gamma}{}^{\gamma'}h^{\eta}{}_{\eta'}
         \nabla_{\gamma'}\nabla^{\eta'}\beta\left(\phi\right)
      \right)
      \nonumber\\&\qquad\qquad\qquad\qquad
      +
      \left(
         h^{\rho}{}_{\rho'}h_{\mu}{}^{\mu'}h_{\nu}{}^{\nu'}
         R^{\rho'\sigma'}{}_{\mu'\nu'}
      \right)
      \left(
         -h_{\gamma}{}^{\gamma'}n^{\eta}n_{\eta'}
         \nabla_{\gamma'}\nabla^{\eta'}\beta\left(\phi\right)
      \right)
   \Big)
   \nonumber\\
   =&
   .
\end{align}
Putting everything together, we see that the Hamiltonian
and momentum constraints reduce to
\begin{subequations}
\label{eqs:full_3p1_constraints}
\begin{align}
   \mathcal{H}
   =&
   \frac{1}{2}\left(
      {}^{(3)}R
      +
      K^2
      -
      K_{ij}K^{ij}
   \right)
   -
   2\delta^{ijk}_{pqr}\left(
      {}^{(3)}R^{pq}{}_{ij}
      +
      2K^p{}_iK^q{}_j
   \right)
   \left(
      D^rD_k\beta\left(\phi\right)
      +
      K^r{}_kB\left(\phi\right)
   \right)
   -
   E
   =
   0
   ,\\
   \mathcal{M}^k
   =&
   -
   D_j\left(
      K^{kj}
      -
      h^{kj}K
   \right)
   \nonumber\\&
   +
   \cdots
   \nonumber\\&
   +
   p^i
   =
   0
   .
\end{align}
\end{subequations}

From Eqs.~\eqref{eqs:full_3p1_constraints}
we see that the Hamiltonian and momentum constraint
equations for $4\partial ST$ gravity
are fundamentally ``elliptic'' type equations; i.e.
they do not involve second time derivatives of either the metric
or scalar field.
%=============================================================================
\subsection{Conformally flat, time symmetric initial data}
\label{eq:cfts_id}
We consider initial data of the form:
\begin{align}
   h_{ij}
   =
   \psi^4\delta_{ij}
   ,\qquad
   K_{ij}
   =
   0
   ,\qquad
   B\left(\phi\right)
   =
   0
   ,
\end{align}
where $\delta_{ij}$ is the flat metric.
With this initial data the momentum constraint is
trivially satisfied (including $p^i=0$),
so all we need to do is solve
for the conformal factor $\psi$ in the Hamiltonian constraint.

The Hamiltonian constraint is
\begin{align}
   \mathcal{H}
   =&
   \frac{1}{2}{}^{(3)}R
   -
   2\delta^{mng}_{rse}
   \left({}^{(3)}R^{rs}{}_{mn}\right)
   \left(D^{e}D_{g}\beta\left(\phi\right)\right)
   -
   E
   \nonumber\\
   =&
   -
   4\frac{1}{\psi^5}\partial_i\partial^i\psi
   +
   8\delta^{mg}_{se}
   \left(\partial^e\partial_g\beta\left(\phi\right)\right)
   \frac{1}{\psi}\left(\partial_m\partial^s\psi\right)
   \nonumber\\
   &
   -
   8\delta^{mg}_{se}
   \left(\partial^e\partial_g\beta\left(\phi\right)\right)
   \frac{1}{\psi^2}\left(\partial_m\psi \partial^s\psi\right)
   -
   4\left(
      f^{qs}\delta^{mng}_{rse}
      \partial^e\partial_g\beta\left(\phi\right)
   \right)
   P^r_{ml}P^l_{qn}
   -
   E
   =
   0
   ,
\end{align}
With our initial data choice we have
\begin{align}
   X
   =
   -
   \frac{1}{2}\frac{1}{\psi^4}\delta^{ij}\partial_i\phi \partial_j\phi
   .
\end{align}
Multiplying the Hamiltonian constraint through
by $-\psi^5/8$, the principle symbol of this system is
\begin{align}
   \mathcal{P}^{(\mathcal{H})}\left(\xi\right)
   =
   \left(
      \left(
         1
         -
         2\psi^4\partial_kD^k\beta\left(\phi\right)
      \right)
      f^{ij}
      +
      2\psi^4D^iD^j\beta\left(\phi\right)
   \right)
   \xi_i\xi_j 
   .
\end{align}
As long as $\psi^4D_iD_j\beta\left(\phi\right)$ is small enough,
we should expect the Hamiltonian constraint to remain an elliptic
equation for $\psi$.
%=============================================================================
\subsection{Conformal thin sandwich}
\label{eq:cts}
We set
\begin{align}
   h_{ij}
   =
   \psi^4\delta_{ij}
   ,\qquad
   K_{ij}
   =
   \frac{1}{\psi^2}\left(
      \partial_iV_j
      +
      \partial_jV_i
      -
      \frac{2}{3}\delta_{ij}\partial_kV^k
   \right)
   .
\end{align}
%=============================================================================
\section{Useful identities}
\label{sec:useful_identities}

Some useful identities are
\begin{subequations}
\begin{align}
   h_{\alpha}{}^{\alpha'}h_{\beta}{}^{\beta'}
   h_{\gamma}{}^{\gamma'}h_{\delta}{}^{\delta'}
   R_{\alpha'\beta'\gamma'\delta'}
   =&
   {}^{(3)}R_{\alpha\beta\gamma\delta}
   +
   K_{\alpha\gamma}K_{\beta\delta}
   -
   K_{\alpha\delta}K_{\beta\gamma}
   ,\\
   h_{\alpha}{}^{\alpha'}h_{\beta}{}^{\beta'}
   h_{\gamma}{}^{\gamma'}n^{\delta'}
   R_{\alpha'\beta'\gamma'\delta'}
   =&
   D_{\beta}K_{\alpha\gamma}
   -
   D_{\alpha}K_{\beta\gamma}
   ,\\
   h_{\alpha}{}^{\alpha'}h_{\gamma}{}^{\gamma'}
   n^{\beta'}n^{\delta'}
   R_{\alpha'\beta'\gamma'\delta'}
   =&
   \mathcal{L}_nK_{\alpha\gamma}
   +
   \frac{1}{N}D_{\alpha}D_{\gamma}N
   +
   K_{\alpha\beta}K^{\beta}{}_{\gamma}
   ,\\
   h_{\alpha}{}^{\alpha'}h_{\beta}{}^{\beta'}
   \nabla_{\alpha'}\nabla_{\beta'}\phi
   =&
   D_{\alpha}D_{\beta}\phi
   +
   K_{\alpha\beta}\mathcal{L}_n\phi
   ,\\
   h_{\alpha}{}^{\alpha'}n^{\beta'}
   \nabla_{\alpha'}\nabla_{\beta'}\phi
   =&
   D_{\alpha}\left(\mathcal{L}_n\phi\right)
   +
   K_{\alpha}{}^{\beta}D_{\beta}\phi
   ,\\
   n^{\alpha'}n^{\beta'}
   \nabla_{\alpha'}\nabla_{\beta'}\phi
   =&
   \mathcal{L}_n\left(\mathcal{L}_n\phi\right)
   -
   a^{\alpha}D_{\alpha}\phi
   .
\end{align}
\end{subequations}
%=============================================================================
\section{Spatial conformal transformation}
Under the conformal transformation
\begin{subequations}
\begin{align}
   h_{ij}
   =&
   \psi^4\tilde{h}_{ij}
   ,\\
   K_{ij}
   =&
   \psi^{-2}\tilde{A}_{ij}
   +
   \frac{1}{3}K h_{ij}
   ,\\
   K_{\phi}
   =&
   \psi^{-6}\tilde{K}_{\phi}
   ,
\end{align}
\end{subequations}
we have
\begin{subequations}
\begin{align}
   {}^{(3)}\Gamma^k_{ij}
   =&
   {}^{(3)}\tilde{\Gamma}^k_{ij}
   +
   2\left(
      \delta^k_i\tilde{D}_j\ln\psi
      +
      \delta^k_j\tilde{D}_i\ln\psi
      -
      \tilde{h}_{ij}\tilde{D}^k\ln\psi
   \right)
   \equiv
   {}^{(3)}\tilde{\Gamma}^k_{ij}
   +
   \mathcal{P}^k_{ij}
   ,\\
   D_iD_j\phi
   =&
   \tilde{D}_i\tilde{D}_j\phi
   -
   \mathcal{P}^k_{ij}\tilde{D}_k\phi
   ,\\
   {}^{(3)}R^{ik}{}_{jl}
   =&
   \psi^{-4}{}^{(3)}\tilde{R}^{ik}{}_{jl}
   -
   8\psi^{-5}\delta^{[i}_{[j}\tilde{D}^{k]}\tilde{D}_{l]}\psi
   +
   24\psi^{-6}\delta^{[i}_{[j}\tilde{D}^{k]}\psi\tilde{D}_{l]}\psi
   -
   4\psi^{-6}\delta^{ik}_{jl}\left(\tilde{D}\psi\right)^2
   ,\\
   D^jK_i^k
   =&
   \psi^{-10}\tilde{D}^j\tilde{A}_i^k
   +
   \frac{1}{3}\psi^{-4}\tilde{D}^j\psi\delta^k_i
   -
   6\psi^{-10}\tilde{D}^j\tilde{A}^k_i
   \nonumber\\
   &
   +
   2\psi^{-10}\left(
      \tilde{h}^{jk}\tilde{A}^p_i\tilde{D}_p\ln\psi
      +  
      \delta^j_i\tilde{A}^{kp}\tilde{D}_p\ln\psi
      -
      \tilde{A}^j_i\tilde{D}^k\ln\psi
      -
      \tilde{A}^{jk}\tilde{D}_i\ln\psi
   \right)
   .
\end{align}
\end{subequations}
%=============================================================================
\end{document}

