%****** Start of file apssamp.tex ******
%
%   This file is part of the APS files in the REVTeX 4.1 distribution.
%   Version 4.1r of REVTeX, August 2010
%
%   Copyright (c) 2009, 2010 The American Physical Society.
%
%   See the REVTeX 4 README file for restrictions and more information.
%
% TeX'ing this file requires that you have AMS-LaTeX 2.0 installed
% as well as the rest of the prerequisites for REVTeX 4.1
%
% See the REVTeX 4 README file
% It also requires running BibTeX. The commands are as follows:
%
%  1)  latex apssamp.tex
%  2)  bibtex apssamp
%  3)  latex apssamp.tex
%  4)  latex apssamp.tex
%
\documentclass[%
%reprint,
%preprint,
notitlepage,
%superscriptaddress,
%groupedaddress,
%unsortedaddress,
%runinaddress,
%frontmatterverbose, 
report,
%showpacs,preprintnumbers,
nofootinbib,
%nobibnotes,
%bibnotes,
 amsmath,amssymb,
 aps,
%pra,
%prb,
%rmp,
%prstab,
%prstper,
%floatfix,
%]{revtex4}
]{revtex4-1}

\usepackage{graphicx}
%\graphicspath{{.}{jripley_plots/}}
% Include figure files
%\usepackage{dcolumn}% Align table columns on decimal point
\usepackage{float}% to force figure placement 
\usepackage{bm}% bold math
\usepackage[hidelinks]{hyperref}
%\usepackage[mathlines]{lineno}% Enable numbering of text and display math
%\linenumbers\relax % Commence numbering lines
\usepackage{color}

\usepackage[utf8]{inputenc}
\usepackage[T1]{fontenc}
\usepackage{slashed}

%\usepackage{fullpage}
\usepackage{enumerate}

\usepackage{mathalfa,amssymb,amsthm}
\usepackage{amsmath}

%\usepackage{subfig}

\newcommand{\we}[1]{\textcolor{blue}{WE: #1}}
\newcommand{\jr}[1]{\textcolor{green}{JR: #1}}

\newcommand{\ssec}[1]{\section{#1}}

\begin{document}

%\preprint{APS/123-QED}
%=============================================================================
\title{Two punctures code adapted for (modified) generalized harmonic evolution}
%\thanks{A footnote to the article title}%
%-----------------------------------------------------------------------------
\author{Justin L. Ripley}
\email{lloydripley@gmail.com}
\affiliation{%
DAMTP,
Centre for Mathematical Sciences,
University of Cambridge,
Wilberforce Road, Cambridge CB3 0WA, UK.
}%
%=============================================================================
\begin{abstract}
   We describe the use of multiple black hole pucture-type initial data
   for evolution using the generalized harmonic formulation of the
   Einstein equations.
   Each section can essentially be read independently, in no particular order.
   We use units with $G=c=1$, with $-+++$ metric signature, and
   $R^{\mu}{}_{\alpha\nu\beta}=\partial_{\nu}\Gamma^{\mu}_{\alpha\beta}-\cdots$.
   Lower-case Greek letters label spacetime indices, while lower-case
   Latin letter label spatial indices.
\end{abstract}
%=============================================================================
\maketitle
%=============================================================================
\ssec{The code}%
The code computes puncture initial data for two black holes, which
we label $1$ and $2$.
The freely specifiable variables are the puncture locations,
momenta, and spins, the lapse and shift, and the gauge source terms
$H^0$ and $H^i$:
\begin{align}
   P_{(1)}^i, S_{(1)}^i, P_{(2)}^i, S_{(2)}^i, \alpha, \beta^i, H^0, H^i
   .
\end{align}
The code solves the momentum constraint to determine $\gamma_{ij}$.
With that, we can determine $g_{\mu\nu}$ and $\partial_{\alpha}g_{\mu\nu}$
on the initial data surface.
%=============================================================================
\ssec{Puncture initial data}%
Nontrivial puncture data for multiple black hole spacetimes was first
introduced by Bowen and York\cite{Bowen:1980yu}, and was later
improved for numerical implementation by
Brandt and Bruegmann\cite{Brandt:1997tf}.
We briefly review the construction here.

We consider vacuum General Relativity (GR). The Hamiltonian and
momentum constraints are
\begin{subequations}
\begin{align}
   {}^{(3)}R+K^2-K_{ij}K^{ij} 
   =&
   0
   ,\\
   D_j\left(K^{ij}-\gamma^{ij}K\right)
   =&
   0
   ,
\end{align}
\end{subequations}
where $\gamma_{ij}$ is the 3-metric, $K_{ij}$ is the extrinsic curvature,
$K=\gamma^{ij}K_{ij}$, and $D$ and ${}^{(3)}R$ are the covariant derivative
and Ricci scalar curvature associated with $\gamma_{ij}$.

We consider initial data of the form
\begin{subequations}
\begin{align}
   \gamma_{ij}
   =&
   \psi^4\delta_{ij}
   ,\\
   K_{ij}
   =&
   \psi^{-2}
   \tilde{A}_{ij}
   ,\\
   \tilde{A}_{ij}
   \equiv&
   \left(
      \partial_iV_j
   +  \partial_jV_i
   -  \frac{2}{3}\delta_{ij}\partial_kV^k
   \right)
   .
\end{align}
\end{subequations}
This initial data is (spatially)
conformally flat and maximally sliced ($K=0$).
The Hamiltonian and momentum constraints reduce to
\begin{align}
   \Delta \psi
   +  \frac{1}{8}\psi^{-7}\tilde{A}_{ij}\tilde{A}^{ij}
   =&
   0
   ,\\
   \Delta V^k
+  \frac{1}{3}\partial^k \left(\partial_iV^i\right)
   =&
   0
   .
\end{align}
   We can find an analytic solution for the vector $V^i$: 
\begin{align}
\label{eq:Bowen_York_sol}
   V^i
   =
   \sum_{n=1}^N
   \left(
   -  \frac{1}{4|x_{(n)}|}\left(
         7P_{(n)}^i
      +  \hat{x}^j_{(n)} P_{j(n)}\hat{x}_{(n)}^i
      \right)
   +  \frac{1}{| x_{(n)} |^2}
      \epsilon^{ijk}\hat{x}_{j(n)}S_{k(n)}
   \right)
   .
\end{align} 
We plug in the following ansatz for the conformal factor
\begin{align}
   \psi
   =
   1
   +  \frac{1}{\zeta}
   +  u
   ,\qquad
   \frac{1}{\zeta}
   \equiv
+  \sum_{n=1}^N\frac{M_{(n)}}{2|x_{(n)}|}
   ,
\end{align} 
into the Hamiltonian constraint to obtain a nonlinear elliptic
equation for the variable $u$:
\begin{align}
\label{eq:u_eqn}
   \Delta u 
   + 
   \frac{1}{8}\zeta^7\tilde{A}_{ij}\tilde{A}^{ij}
   \left(1+\zeta+\zeta u\right)^{-7}
   =
   0
   .
\end{align}
From this equation we can conclude that $u$ is $C^{\infty}$ everywhere
in cartesian coordinates, except at the punctures, where it is only $C^2$.
Unlike the momentum constraint, there is no known analytic
solution to Eq.~\eqref{eq:u_eqn}, so it needs to be solved numerically.

In summary, with puncture initial data we can freely specify
$K_{ij}$ via the Bown-York solution \cite{eq:Bowen_York_sol},
and we can freely specify the lapse $\alpha$ and shift $\beta^i$.
The constrained variables are $\gamma_{ij}$, which is
conformally flat, and $\partial_t\gamma_{ij}$.
We find $\gamma_{ij}$ with the Hamiltonian constraint,
and $\partial_t\gamma_{ij}$ is related to the free initial data via
\begin{align}
   \partial_t\gamma_{ij}
   =
   -
   2\alpha K_{ij}
   +
   D_i\beta_j
   +
   D_j\beta_i
   .
\end{align}
%=============================================================================
\section{Relating puncture initial data to four dimensional initial data}
We want initial data for the 20 quantities
\begin{align}
   g_{\mu\nu},\qquad\partial_tg_{\mu\nu}
   .
\end{align}
The metric is related to the ADM variables by:
\begin{subequations}
\begin{align}
   g_{00}
   =
   -  
   \alpha^2
   +
   \gamma_{ij}\beta^i\beta^j
   ,\qquad
   g_{0i}
   =
   \beta_i
   ,\qquad
   g_{ij}
   =
   \gamma_{ij}
   .
\end{align}
\end{subequations}
Puncture initial data gives us:
\begin{align}
   \gamma_{ij},\qquad K_{ij}
   .
\end{align}
but does not specify the lapse and shift, so we can freely pick those.
One we have chosen a lapse and shift, we can compute the following derivatives:
\begin{subequations}
\begin{align}
   \partial_t\gamma_{ij}
   =
   -
   2\alpha K_{ij}
   +
   D_i\beta_j
   +
   D_j\beta_i
   ,\qquad
   \partial_k\gamma_{ij}
   ,\qquad
   \partial_k\alpha
   ,\qquad
   \partial_k\beta^i
   .
\end{align}
\end{subequations}
All we need are $\partial_t\alpha$ and $\partial_t\beta^i$.
We specify these with the gauge source functions via
Eqs.~\eqref{eq:GH_evo_lapse_shift}.

If we use $\beta^i=K=0$, the relevant equations are:
\begin{subequations}
\begin{align}
   \partial_t\gamma_{ij}
   =&
   -
   2\alpha K_{ij}
   ,\\
   \partial_t\alpha
   =&
   \alpha^3H^0
   ,\\
   \partial_t\beta^i
   =&
   -  
   \alpha\gamma^{ij}\partial_j\alpha
   +  
   \alpha^2\left({}^{(3)}\Gamma^i\right)
   +  
   \alpha^2H^i
   ,\\
   g_{00}
   =&
   -
   \alpha^2
   ,\\
   g_{0i}
   =&
   0
   ,\\
   g_{ij}
   =&
   \gamma_{ij}
   .
\end{align}   
\end{subequations}
%=============================================================================
\section{$1+3$ spacetime slicing and $ADM$ variables}
\label{sec:eom_lapse_shift}
   We foliate our spacetime with spacelike hypersurfaces
$\Sigma_{(t)}$, which have unit timelike normal covector $n_{\alpha}$
\begin{align}
   g_{\alpha\beta}
   =
-  n_{\alpha}n_{\beta}
+  \gamma_{\alpha\beta}
   .
\end{align}
   By construction $n^{\alpha}\gamma_{\alpha\beta}=0$.
The extrinsic curvature is 
\begin{align}
   K_{\alpha\beta}
   \equiv
-  \gamma_{\alpha}{}^{\mu}
   \gamma_{\beta}{}^{\nu}
   \nabla_{\mu}n_{\nu}
   .
\end{align}
   Traditionally gauge conditions are set using the $1+3$ dimensional ADM
variables. We will be deriving the equations of motion for the lapse and
shift the MGH formulation. 
We set $n_{\alpha}=\left(-\alpha,0,0,0\right)$, where
$\alpha$ is the lapse function.
We adapt coordinates to the spacelike foliation
\begin{align}
   ds^2
   =
-  \alpha^2dt^2
+  \gamma_{ij}\left(dx^i+\beta^idt\right)\left(dx^j+\beta^jdt\right)
   ,
\end{align}
   where $\beta^j$ is the shift vector.
The intrinsic metric is $\gamma_{ij}$, and the extrinsic curvature is
\begin{align}
   K_{ij}
   =&
-  \frac{1}{2\alpha}\left(
      \partial_t\gamma_{ij}
   -  D_i\beta_j
   -  D_j\beta_i
   \right) 
   .
\end{align}
   Some useful relations are
(see also Appendix B of \cite{alcubierre2008introduction})
\begin{subequations}
\begin{align}
   \Gamma^0
   =&
-  \frac{1}{\alpha^3}\left(
      \partial_t
   -  \beta^i\partial_i
   \right)
   \alpha
-  \frac{1}{\alpha}K
   ,\\
   \Gamma^i
   =&
-  \frac{1}{\alpha^2}\left(
      \partial_t
   -  \beta^j\partial_j
   \right)
   \beta^i
-  \gamma^{ij}\frac{1}{\alpha}\partial_j\alpha
+  {}^{(3)}\Gamma^i
-  \beta^i\Gamma^0
   ,\\
   n^{\alpha}n^{\beta}\Gamma^0_{\alpha\beta}
   =&
   \frac{1}{\alpha^3}\left(
      \partial_t
   -  \beta^i\partial_i
   \right)
   \alpha
   ,\\
   n^{\alpha}n^{\beta}\Gamma^i_{\alpha\beta}
   =&
   \frac{1}{\alpha^2}\left(
      \partial_t
   -  \beta^j\partial_j
   \right)\beta^i
-  \frac{\beta^i}{\alpha^3}
   \left(
      \partial_t
   -  \beta^j\partial_j
   \right)
   \alpha
+  \frac{1}{\alpha}\gamma^{ij}\partial_j\alpha
   ,\nonumber\\
   =&
   \frac{1}{\alpha^2}\left(
      \partial_t
   -  \beta^j\partial_j
   \right)\beta^i
+  \beta^i\Gamma^0
+  \frac{\beta^i}{\alpha}K
+  \frac{1}{\alpha}\gamma^{ij}\partial_j\alpha
   .
\end{align}
\end{subequations}
   The following identities are helpful in deriving the above equations:
\begin{subequations}
\begin{align}
   \Gamma^{\alpha}
   =&
-  \frac{1}{|g|^{1/2}}\partial_{\beta}
   \left(
      |g|^{1/2}g^{\alpha\beta}
   \right)
   ,\\
   n^{\alpha}n^{\beta}\Gamma^{\gamma}_{\alpha\beta}
   =&
-  n^{\alpha}n_{\delta}\partial_{\alpha}g^{\gamma\delta}
-  g^{\gamma\delta}n^{\alpha}\partial_{\delta}n_{\alpha}
   .
\end{align}
\end{subequations}
%=============================================================================
\section{Generalized Harmonic (GH) formulation}
In the GH formulation we set
\begin{align}
\label{eq:gh_condition}
   C^{\gamma}
   \equiv&
   -\Box x^{\gamma}
+  H^{\gamma}
   \nonumber\\
   =& 
   \Gamma^{\gamma}
+  H^{\gamma} = 0
   .
\end{align} 
We have defined the \emph{constraint violation}
$C^{\gamma}$ (which will generally not be exactly zero in a given numerical solution)
, and the \emph{source function} $H^{\alpha}$.
We can specify the source functions however we wish.
For example, Lindblom et. al. \cite{Lindblom:2005qh} specify
$H^{\alpha}$ with an algebraic condition,
and Pretorius makes $H^{\alpha}$
satisfy a wave equation \cite{Pretorius:2004jg}.
Once we have specified $H^{\alpha}$, we have picked a gauge.

   We next define the GH equations of motion (EOM) as
\begin{align}
\label{eq:GH_tensor_eom}
   &
   E^{\alpha\beta} 
-  P_{\delta}{}^{\gamma\alpha\beta}\nabla_{\gamma}C^{\delta}
-  \frac{1}{2}\kappa
   \left(
      n^{\alpha}C^{\beta}
   +  n^{\beta}C^{\alpha}
   +  \left(1+\rho\right)n^{\gamma}C_{\gamma}g^{\alpha\beta}
   \right)
   =
   0
   ,
\end{align}
   where $E^{\alpha\beta}$ are the EOM derived from varying the metric
(e.g. for the Einstein equations we'd have 
$E^{\alpha\beta}=R^{\alpha\beta}-\frac{1}{2}g_{\alpha\beta}R$),
$n^{\alpha}$ is the unit time-like vector orthogonal to the spatial foliation
for our spacetime, and
\begin{align}
   P_{\delta}{}^{\gamma\alpha\beta}
   \equiv
   \frac{1}{2}
   \left(
      \delta_{\delta}^{\alpha}g^{\beta\gamma}
   +  \delta_{\delta}^{\beta}g^{\alpha\gamma}
   -  \delta_{\delta}^{\gamma}g^{\alpha\beta}
   \right)
   .
\end{align}
   We have included constraint damping with the
constants $\kappa$ and $\rho$ \cite{Gundlach:2005eh}
(Note that we need $\kappa<0$ to damp out the constraints).
As most gauge conditions are specified in terms of the lapse and shift,
we rewrite Eq.~\eqref{eq:gh_condition} in terms of those variables:
\begin{subequations}
\label{eq:GH_evo_lapse_shift}
\begin{align}
   \left(
      \partial_t
   -  \beta^i\partial_i
   \right)
   \alpha
   +  \alpha^2K
   -  \alpha^3H^0
   =&
   0
   ,\\
   \left(
      \partial_t
   -  \beta^j\partial_j
   \right)
   \beta^i
+  \alpha\gamma^{ij}\partial_j\alpha
-  \alpha^2\left({}^{(3)}\Gamma^i\right)
-  \alpha^2\beta^iH^0
-  \alpha^2H^i
   =&
   0
   .
\end{align}   
\end{subequations}
   For the second relation we used the the component $\gamma=0$ relation
to eliminate the time derivative of the lapse.
In the GH formulation, picking a gauge means picking
a choice of $H^0$ and $H^i$, which we see allows us to specify
$\partial_t\alpha$ and $\partial_t\beta^i$.
%=============================================================================
\bibliography{thebib}
%=============================================================================
\end{document}

